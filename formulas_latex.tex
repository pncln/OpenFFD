% ================================================================================
% FFD Y-DIRECTION OPTIMIZATION: MATHEMATICAL FORMULATION
% Complete LaTeX equations for academic research paper
% ================================================================================

\documentclass{article}
\usepackage{amsmath}
\usepackage{amssymb}
\usepackage{bm}

\begin{document}

% ================================================================================
\section{Optimization Problem Formulation}
% ================================================================================

\subsection{Objective Function}
\begin{equation}
\min_{\mathbf{u}} \quad f(\mathbf{u}) = C_d(\mathbf{u})
\end{equation}

where $\mathbf{u} = [u_1, u_2, u_3, u_4]^T$ are the FFD control point Y-displacements and $C_d$ is the drag coefficient.

\subsection{Design Variable Bounds}
\begin{equation}
\mathbf{u}^L \leq \mathbf{u} \leq \mathbf{u}^U
\end{equation}

where $\mathbf{u}^L = [-0.2, -0.2, -0.2, -0.2]^T$ and $\mathbf{u}^U = [0.2, 0.2, 0.2, 0.2]^T$.

% ================================================================================
\section{FFD Parameterization}
% ================================================================================

\subsection{FFD Control Point Grid}
The FFD control volume is defined by a $[2 \times 2 \times 1]$ grid with control points:
\begin{equation}
\mathbf{P}_{i,j,k} = \mathbf{P}_{i,j,k}^0 + \delta \mathbf{P}_{i,j,k}
\end{equation}

where $\mathbf{P}_{i,j,k}^0$ are the initial control point positions and $\delta \mathbf{P}_{i,j,k}$ are the displacements.

\subsection{Y-Direction Only Constraint}
\begin{equation}
\delta \mathbf{P}_{i,j,k} = \begin{bmatrix} 0 \\ u_{ij} \\ 0 \end{bmatrix}
\end{equation}

where $u_{ij}$ corresponds to the design variables:
\begin{align}
u_{00} &= u_1 \quad \text{(Front Upper)} \\
u_{10} &= u_2 \quad \text{(Rear Upper)} \\
u_{01} &= u_3 \quad \text{(Front Lower)} \\
u_{11} &= u_4 \quad \text{(Rear Lower)}
\end{align}

\subsection{Trilinear Interpolation}
For any point $\mathbf{x}$ in the FFD volume, the displacement is computed using:
\begin{equation}
\delta \mathbf{x} = \sum_{i=0}^{1} \sum_{j=0}^{1} \sum_{k=0}^{0} B_i(\xi) B_j(\eta) B_k(\zeta) \delta \mathbf{P}_{i,j,k}
\end{equation}

where $(\xi, \eta, \zeta)$ are the local FFD coordinates and $B_i$, $B_j$, $B_k$ are Bernstein polynomials:
\begin{align}
B_0(\xi) &= 1 - \xi \\
B_1(\xi) &= \xi \\
B_0(\eta) &= 1 - \eta \\
B_1(\eta) &= \eta \\
B_0(\zeta) &= 1
\end{align}

\subsection{Mesh Point Deformation}
The new mesh coordinates are:
\begin{equation}
\mathbf{x}^{new} = \mathbf{x}^{old} + \delta \mathbf{x}
\end{equation}

% ================================================================================
\section{Governing Equations}
% ================================================================================

\subsection{RANS Equations}
\subsubsection{Continuity Equation}
\begin{equation}
\frac{\partial \rho}{\partial t} + \frac{\partial}{\partial x_i}(\rho u_i) = 0
\end{equation}

\subsubsection{Momentum Equation}
\begin{equation}
\frac{\partial}{\partial t}(\rho u_i) + \frac{\partial}{\partial x_j}(\rho u_i u_j) = -\frac{\partial p}{\partial x_i} + \frac{\partial}{\partial x_j}\left[\mu \left(\frac{\partial u_i}{\partial x_j} + \frac{\partial u_j}{\partial x_i} - \frac{2}{3}\delta_{ij}\frac{\partial u_k}{\partial x_k}\right)\right] + \frac{\partial}{\partial x_j}(-\rho \overline{u_i' u_j'})
\end{equation}

\subsubsection{Turbulent Kinetic Energy ($k$)}
\begin{equation}
\frac{\partial}{\partial t}(\rho k) + \frac{\partial}{\partial x_i}(\rho k u_i) = \frac{\partial}{\partial x_j}\left[\left(\mu + \frac{\mu_t}{\sigma_k}\right) \frac{\partial k}{\partial x_j}\right] + G_k - Y_k
\end{equation}

\subsubsection{Specific Dissipation Rate ($\omega$)}
\begin{equation}
\frac{\partial}{\partial t}(\rho \omega) + \frac{\partial}{\partial x_i}(\rho \omega u_i) = \frac{\partial}{\partial x_j}\left[\left(\mu + \frac{\mu_t}{\sigma_\omega}\right) \frac{\partial \omega}{\partial x_j}\right] + G_\omega - Y_\omega + D_\omega
\end{equation}

\subsection{Turbulent Viscosity}
\begin{equation}
\mu_t = \rho \frac{k}{\omega} \frac{1}{\max\left[\frac{1}{\alpha^*}, \frac{SF_2}{a_1\omega}\right]}
\end{equation}

\subsection{Boundary Conditions}
\subsubsection{Inlet}
\begin{align}
\mathbf{u} &= \mathbf{u}_{inlet} = [30, 0, 0]^T \, \text{m/s} \\
\frac{\partial p}{\partial n} &= 0 \\
k &= k_{inlet} = 0.06 \, \text{m}^2/\text{s}^2 \\
\omega &= \omega_{inlet} = 400 \, \text{s}^{-1} \\
\nu_t &= \nu_{t,inlet} = 1 \times 10^{-5} \, \text{m}^2/\text{s}
\end{align}

\subsubsection{Outlet}
\begin{align}
\frac{\partial \mathbf{u}}{\partial n} &= 0 \\
p &= p_{ref} = 0 \, \text{Pa} \\
\frac{\partial k}{\partial n} &= 0 \\
\frac{\partial \omega}{\partial n} &= 0 \\
\frac{\partial \nu_t}{\partial n} &= 0
\end{align}

\subsubsection{Walls (Airfoil)}
\begin{align}
\mathbf{u} &= \mathbf{0} \quad \text{(no-slip)} \\
\frac{\partial p}{\partial n} &= 0 \\
k &= k_{wall} \quad \text{(kqRWallFunction)} \\
\omega &= \omega_{wall} \quad \text{(omegaWallFunction)} \\
\nu_t &= \nu_{t,wall} \quad \text{(nutkWallFunction)}
\end{align}

% ================================================================================
\section{Force Coefficient Calculation}
% ================================================================================

\subsection{Pressure and Viscous Forces}
\begin{equation}
\mathbf{F} = \int_{\Gamma_{wall}} \left(p \mathbf{n} + \boldsymbol{\tau} \cdot \mathbf{n}\right) dS
\end{equation}

where $\boldsymbol{\tau}$ is the viscous stress tensor:
\begin{equation}
\boldsymbol{\tau} = \mu \left(\nabla \mathbf{u} + (\nabla \mathbf{u})^T - \frac{2}{3}(\nabla \cdot \mathbf{u})\mathbf{I}\right)
\end{equation}

\subsection{Drag Coefficient}
\begin{equation}
C_d = \frac{F_d}{\frac{1}{2}\rho_\infty U_\infty^2 S_{ref}}
\end{equation}

where:
\begin{align}
F_d &= \mathbf{F} \cdot \mathbf{e}_x \quad \text{(drag force)} \\
\rho_\infty &= 1.225 \, \text{kg/m}^3 \quad \text{(reference density)} \\
U_\infty &= 30 \, \text{m/s} \quad \text{(reference velocity)} \\
S_{ref} &= c = 35.0492 \, \text{m} \quad \text{(reference chord length)}
\end{align}

% ================================================================================
\section{Gradient Computation}
% ================================================================================

\subsection{Forward Finite Difference Formula}
\begin{equation}
\frac{\partial f}{\partial u_i} \approx \frac{f(\mathbf{u} + h \mathbf{e}_i) - f(\mathbf{u})}{h}
\end{equation}

where:
\begin{align}
h &= 1 \times 10^{-6} \quad \text{(step size)} \\
\mathbf{e}_i &= \text{unit vector in direction of design variable } i \\
f(\mathbf{u}) &= C_d(\mathbf{u}) \quad \text{(objective function)}
\end{align}

\subsection{Gradient Vector}
\begin{equation}
\nabla f = \begin{bmatrix}
\frac{\partial f}{\partial u_1} \\
\frac{\partial f}{\partial u_2} \\
\frac{\partial f}{\partial u_3} \\
\frac{\partial f}{\partial u_4}
\end{bmatrix}
\end{equation}

\subsection{Computational Cost}
For each gradient evaluation:
\begin{equation}
N_{CFD} = 1 + n_{dv} = 1 + 4 = 5 \text{ CFD simulations}
\end{equation}

where $n_{dv} = 4$ is the number of design variables.

% ================================================================================
\section{SLSQP Optimization Algorithm}
% ================================================================================

\subsection{Sequential Quadratic Programming}
At each iteration $k$, solve the quadratic subproblem:
\begin{align}
\min_{\mathbf{p}} \quad &\nabla f(\mathbf{x}^k)^T \mathbf{p} + \frac{1}{2} \mathbf{p}^T \mathbf{B}^k \mathbf{p} \\
\text{subject to} \quad &\mathbf{x}^L \leq \mathbf{x}^k + \mathbf{p} \leq \mathbf{x}^U
\end{align}

where $\mathbf{B}^k$ is the BFGS approximation of the Hessian.

\subsection{BFGS Hessian Update}
\begin{equation}
\mathbf{B}^{k+1} = \mathbf{B}^k + \frac{\mathbf{y}^k (\mathbf{y}^k)^T}{(\mathbf{y}^k)^T \mathbf{s}^k} - \frac{\mathbf{B}^k \mathbf{s}^k (\mathbf{s}^k)^T \mathbf{B}^k}{(\mathbf{s}^k)^T \mathbf{B}^k \mathbf{s}^k}
\end{equation}

where:
\begin{align}
\mathbf{s}^k &= \mathbf{x}^{k+1} - \mathbf{x}^k \\
\mathbf{y}^k &= \nabla f(\mathbf{x}^{k+1}) - \nabla f(\mathbf{x}^k)
\end{align}

\subsection{Line Search}
Find step size $\alpha^k$ satisfying the Armijo condition:
\begin{equation}
f(\mathbf{x}^k + \alpha^k \mathbf{p}^k) \leq f(\mathbf{x}^k) + c_1 \alpha^k (\nabla f(\mathbf{x}^k))^T \mathbf{p}^k
\end{equation}

where $c_1 = 10^{-4}$ is the Armijo parameter.

\subsection{Design Variable Update}
\begin{equation}
\mathbf{x}^{k+1} = \mathbf{x}^k + \alpha^k \mathbf{p}^k
\end{equation}

% ================================================================================
\section{Convergence Criteria}
% ================================================================================

\subsection{Gradient Norm Tolerance}
\begin{equation}
\|\nabla f(\mathbf{x}^k)\| < \epsilon_g = 1 \times 10^{-6}
\end{equation}

\subsection{Step Size Tolerance}
\begin{equation}
\|\mathbf{x}^{k+1} - \mathbf{x}^k\| < \epsilon_x = 1 \times 10^{-6}
\end{equation}

\subsection{Objective Function Tolerance}
\begin{equation}
|f(\mathbf{x}^{k+1}) - f(\mathbf{x}^k)| < \epsilon_f = 1 \times 10^{-9}
\end{equation}

\subsection{Maximum Iterations}
\begin{equation}
k \leq k_{max} = 100
\end{equation}

% ================================================================================
\section{Example Iteration Calculation}
% ================================================================================

\subsection{Current State (Iteration $k=5$)}
\begin{align}
\mathbf{x}^5 &= [0.05, -0.02, 0.03, -0.01]^T \\
f(\mathbf{x}^5) &= C_d = 0.012589
\end{align}

\subsection{Gradient Computation}
\begin{align}
\frac{\partial f}{\partial u_1} &= \frac{C_d([0.050001, -0.02, 0.03, -0.01]) - 0.012589}{10^{-6}} = \frac{0.012592 - 0.012589}{10^{-6}} = 3.0 \\
\frac{\partial f}{\partial u_2} &= \frac{C_d([0.05, -0.019999, 0.03, -0.01]) - 0.012589}{10^{-6}} = \frac{0.012585 - 0.012589}{10^{-6}} = -4.0 \\
\frac{\partial f}{\partial u_3} &= \frac{C_d([0.05, -0.02, 0.030001, -0.01]) - 0.012589}{10^{-6}} = \frac{0.012591 - 0.012589}{10^{-6}} = 2.0 \\
\frac{\partial f}{\partial u_4} &= \frac{C_d([0.05, -0.02, 0.03, -0.009999]) - 0.012589}{10^{-6}} = \frac{0.012587 - 0.012589}{10^{-6}} = -2.0
\end{align}

\subsection{Gradient Vector}
\begin{equation}
\nabla f(\mathbf{x}^5) = [3.0, -4.0, 2.0, -2.0]^T
\end{equation}

\subsection{Search Direction}
Solve: $\mathbf{B}^5 \mathbf{p}^5 = -\nabla f(\mathbf{x}^5)$
\begin{equation}
\mathbf{p}^5 = [-0.0024, 0.0032, -0.0016, 0.0016]^T
\end{equation}

\subsection{Line Search Result}
\begin{equation}
\alpha^5 = 0.8
\end{equation}

\subsection{Design Variable Update}
\begin{align}
\mathbf{x}^6 &= \mathbf{x}^5 + \alpha^5 \mathbf{p}^5 \\
&= [0.05, -0.02, 0.03, -0.01]^T + 0.8 \times [-0.0024, 0.0032, -0.0016, 0.0016]^T \\
&= [0.04808, -0.01744, 0.02872, -0.00872]^T
\end{align}

\subsection{Convergence Check}
\begin{align}
\|\nabla f(\mathbf{x}^5)\| &= \sqrt{3.0^2 + (-4.0)^2 + 2.0^2 + (-2.0)^2} = \sqrt{33} = 5.745 \\
\|\mathbf{x}^6 - \mathbf{x}^5\| &= \|[-0.00192, 0.00256, -0.00128, 0.00128]^T\| = 0.00435 \\
|f(\mathbf{x}^6) - f(\mathbf{x}^5)| &\approx 1.3 \times 10^{-5}
\end{align}

Since $\|\nabla f(\mathbf{x}^5)\| = 5.745 > \epsilon_g = 10^{-6}$, continue optimization.

% ================================================================================
\section{FFD Shape Sensitivity}
% ================================================================================

\subsection{Chain Rule for Shape Sensitivity}
\begin{equation}
\frac{\partial C_d}{\partial u_i} = \frac{\partial C_d}{\partial \mathbf{x}} \cdot \frac{\partial \mathbf{x}}{\partial u_i}
\end{equation}

where $\frac{\partial \mathbf{x}}{\partial u_i}$ is the mesh sensitivity and $\frac{\partial C_d}{\partial \mathbf{x}}$ is the flow sensitivity.

\subsection{Mesh Sensitivity (FFD)}
\begin{equation}
\frac{\partial \mathbf{x}}{\partial u_i} = \frac{\partial}{\partial u_i}\left[\mathbf{x}^0 + \sum_{j=0}^{1} \sum_{k=0}^{1} B_j(\eta) B_k(\zeta) \delta \mathbf{P}_{j,k} \right]
\end{equation}

For Y-direction only displacement:
\begin{equation}
\frac{\partial \mathbf{x}}{\partial u_i} = \begin{bmatrix} 0 \\ B_j(\eta) B_k(\zeta) \\ 0 \end{bmatrix}
\end{equation}

where $(j,k)$ corresponds to the control point associated with design variable $u_i$.

\end{document}